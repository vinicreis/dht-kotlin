\documentclass{article}

\usepackage[brazilian]{babel}
\usepackage[letterpaper,top=2cm,bottom=2cm,left=3cm,right=3cm,marginparwidth=1.75cm]{geometry}

\usepackage{amsmath}
\usepackage{graphicx}
\usepackage[colorlinks=true, allcolors=blue]{hyperref}

\title{DHT Service}
\author{Vinícius de Oliveira Campos dos Reis
\and 11041416}
\date{}

\begin{document}
    \maketitle

    \section{O projeto}\label{sec:introducao}

    O projeto é uma implementação em Kotlin que utiliza o Gradle como compilador e gerenciador de dependências.
    A arquitetura de cada componente da aplicação é multi-módulos, seguindo ao máximo uma arquitetura limpa e orientada a domínios.

    \section{Vídeo}\label{sec:video}

    O link para o vídeo de apresentação e demonstração pode ser encontrado \href{}{neste link}.

    \section{Como executar}\label{sec:como-executar}

    Todos módulos neste projeto podem ser executados através do Makefile na raiz do projeto, cada um com seu target.
    Confira abaixo como cada um deles pode ser executado:

    \begin{itemize}
        \item Servidor DHT: \textit{make dht}
        \item Sample - Cofre de senhas: \textit{make sample\_vault}
    \end{itemize}

    \section{Implementação}\label{sec:implementacao}

    \subsection{Servidor}\label{subsec:server}

    A lógica principal do funcionamento do servidor está implementada na classe \textit{DHTServiceServerGrpcImpl}.

    O funcionamento do servidor é baseado no algoritmo de tabelas de hash distribuídas (DHT) em um formato de anel, como proposto para o projeto.

    O contrato implementado por esta classe é muito semelhante ao que foi proposto no enunciado.

    A maior dificuldade encontrada no desenvolvimento foi criar testes unitários para validação e coordenar mais de dois nós na rede.
    Gerenciar diferentes chamadas via gRPC foi um desafio superado, pois os canais que são criados precisam ser fechados após o uso.
    Isso por que estes canais podem ser coletados pelo garbage collector caso fiquem inativos por longos períodos.

    \subsection{Sample - Cofre de Senhas (Key Vault)}\label{subsec:cliente}

    Para demonstração do funcionamento da DHT, foi implementado uma aplicação simples que funciona como um cofre de senhas.

    Esta aplicação implementa um cliente gRPC que se conecta ao servidor DHT e armazena senhas como pares de chave e valor.

    Dado o funcionamento da DHT, a aplicação é capaz de armazenar, deletar e recuperar senhas de forma distribuída.

\end{document}